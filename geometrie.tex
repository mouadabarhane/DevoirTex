\documentclass{exam}
\usepackage[tikz]{
bclogo}
\begin{document}

\begin{center}
\fbox{\fbox{\parbox{5.5in}{\centering
Collège Moulay Bouchta Zghira \ \ \ \ \ \ \ \ \ \ \ \textbf{Devoir Surveillé \ \ \ \ \ \ \ \ \ Durée : 1 Heure\\
Pr : ABARHANE Mouad \ \ Matière : Mathématiques \ \  Année scolaire : 22/23}}}}
\end{center}

\vspace{5mm}
\makebox[0.6\textwidth]{Nom et Prénom:\enspace\hrulefill} \ \ \makebox[0.12\textwidth]{ $ N^{\circ} \ :$\enspace\hrulefill} \ \ \ \  \makebox[0.1\textwidth]{ Classe : 2/..... \enspace\hrulefill} 

\begin{questions}
\question ABC est un triangle , alors : 
\textbf{(4 pts)}
\begin{parts}
\part La droite remarquable issue de A et qui coupe [BC] en son milieu est : \makebox[0.3\textwidth]{\enspace\hrulefill}
\part Le point d'intersection des hauteurs est :
 \makebox[0.3\textwidth]{\enspace\hrulefill}
 \part Le point d'intersection des médianes est :
 \makebox[0.3\textwidth]{\enspace\hrulefill}
  \part La droite remarquable qui partage l'angle  en deux angles égaux est :
 \makebox[0.3\textwidth]{\enspace\hrulefill}
\end{parts}
\question Donner la définition de la médiatrice :
\textbf{(1 pt)} 
\\[0.2cm]
\makebox[1\textwidth]{\enspace\hrulefill}
\question \textbf{Exercice 2 :} 
\textbf{(3 pts)} 
\\[0.2cm]
1)Tracer le triangle ABC , où :\\
\hspace*{0.2cm} a) ABC est un triangle isocèle en A (Figure 1 ). \\
\hspace*{0.2cm} b) ABC est un triangle équilateral (Figure 2 ) .\\
\hspace*{0.2cm} c) ABC est un triangle rectangle en A (Figure 3 ) . \\
2) Déterminer :\\
\hspace*{0.2cm} a) Le centre de Gravité de la Figure 1 . \\
\hspace*{0.2cm} b) L’orthocentre de la Figure 2 . \\
\hspace*{0.2cm} c) L’orthocentre de la Figure 3 .\\
\makebox[1\textwidth]{\enspace\hrulefill}
\question \textbf{Exercice 1 :} 
\textbf{(10 pts)} 
\\[0.2cm]
1) Tracer une droite $ (\Delta) $ .\\[0.2cm]
 Soit A un point du plant , tel que :  $ A \notin (\Delta) $ .\\[0.2cm]
2) Construire la droite $ (D) $ perpendiculaire  à la droite $ (\Delta) $ , et passant par le point A,\underline{en utilisant le compase}.\\
3) Déterminer le point A' le symétrique du point A par rapport  à la droite $ (\Delta) $ .\\[0.2cm]
Choisir le point B , tel que :  $ B \in (\Delta) $ et  $ B \notin [AA'] $  .\\[0.2cm]
4) Quelle est la nature du triangle ABA' ?\\
5)  Tracer la médiatrice du segment [AB] , puis la médiatrice du segment [A'B] .\\
6) Que représente la droite $(\Delta) $ pour le triangle ABA' ? Justifie.\\[0.2cm]
Soit O le point d’intersection obtenu.\\[0.2cm]
7)Tracer le cercle circonscrit du triangle ABA'.\\[0.2cm]
Soit B' est le symétrique du point B par rapport au segment [AA'] .\\[0.2cm]
8) Quelle est la nature du quadrilatère AB'A'B ?
\\[0.2cm]
\makebox[1\textwidth]{\enspace\hrulefill}
\question \textbf{Exercice 3 :} 
\textbf{(2 pts)} 
\\[0.2cm]
Tracer un angle $ \widehat{IOJ} $ de  $ 80^{\circ}$ .\\
Construire sa bissectrice (d) avec la règle et le compas .
\end{questions}

\end{document}